\section*{Solutions}

\aleksandar{Work-in-progress}

Individual materials are segmented from the full image, which is represented by their respective histograms.

\begin{equation}
H(x,v) = \sum_{m=1}^{n} g_{m}(x,v) + r(x,v)
\end{equation}

Where $g_m$ represents a histogram for materials $m\in\{1,n\}$, and $r(x,v)$ is incorrectly identified remainding region.

We find the probability for a certain material as:

\begin{equation}
p(m|x,v) = \frac{g_{m}(x,v)}{H(x,v)}
\end{equation}

We then have a single distribution is given by a piecewise polynomial:

\begin{equation}
g_{m}(x,v) = a_{m}(x) e^{ -b_{m}(x) |v-c_{m}(x)|^{d_{m}(x)} }
\end{equation}

Where a is the height, b is the width, c is the centre and d is the exponent.

If we were to look at the distributions for individual slices, we would get too much variance.
Instead we would like to fit the individual coefficients to get a smooth function.

\subsection*{Miscellaneous}

We shall go through these subjects and explain what was done and how + why. We briefly explain how the method works.

\begin{itemize}
 \item Volume matching
 \item Find implant mask
 \item EDT (Euclidean distance transform)
 \item Gauss
 \item Gauss + EDT
 \item Hist(x,y,z,r)
 \item Find lines / ridges
 \item Get distributions
 \item Beskrive processen under GUI trinvist
 \item Segment materials from distributions
 \item Bayesian combination of materials from their distributions
\end{itemize}

\subsection*{Probabilities}

\aleksandar{Work-in-progress}

Ud fra et 3d-billede (volumen), kan vi tælle os til de enkelte fordelinger

p(I), p(I|x), p(I|y), p(I|z), p(I|r), hvor $r=\sqrt{(x-x_0)^2+(y-y_0)^2}$

ved at finde peaks og fitte distributioner kan vi da finde approksimationer til

p(c|I), p(c|I,x), p(c|I,y), p(c|I,z), p(c|I,r)

hvor c er klasserne, fx knogle, blodåre, væv, ... 

Vi vil gerne finde kombinationsfordelinger som kan klassificere voxels ved brug af mere information end kun intensiteten - blandt andet hvor voxelen befinder sig.

Ud fra fx p(c|I,x) og p(c|I,y) og ... vil vi gerne approksimere p(c|I,x,y,z,r) uden at skulle repræsentere den eksplicit som et kæmpe volumen.

Dette vil vi gøre ved at bruge betinget uafhængiged. \aleksandar{Evt inkludere bevis for at det er validt. Ellers så blot forklare hvordan sandsynlighederne sammensættes og bruges.}
