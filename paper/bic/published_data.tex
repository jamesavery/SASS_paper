\section{Published data}
\label{sec:pubdata}

Generated and simulated volumes are made available for external benchmarking and testing.

\begin{table*}
    \caption{Overview of the published data.}
    \label{tab:pubdata}
    \centering
    \begin{tabular}{@{}p{2cm}p{5cm}p{2.5cm}p{2.5cm}p{2.5cm}@{}}
    %\begin{tabular}{llll}
    \toprule
	    \textbf{Type} & \textbf{Content} & \textbf{File format} & \textbf{Data type} & \textbf{Sizes} \\
    \midrule
	    Masks & Individual volumes for bone, blood, implant, osteocyt & HDF5 & UINT8 & (1/2/4/8)$\times$ \\
	    Mesh models & Individual volumes for bone, blood, implant, osteocyt & STL & N/A & N/A \\
	    Ground truth & Single volume & HDF5 & UINT8 & (1/2/4/8)$\times$ \\
	    Simulated tomograms & Single volume & HDF5 + RAW & FLOAT32 & (2/4/8)$\times$ \\
	    Original tomograms & 4-5 subvolumes per tomogram & HDF5 + RAW & FLOAT32 & (1/2/4/8)$\times$ \\
    \bottomrule
    \end{tabular}
\end{table*}

For all volumes, the first index traverses through the depth of implant, the second index along the short side of the implant, the third and final index along the wide side of the implant. The full-size volumes, corresponding to 1$\times$ scaling, have (3272,3456,3456) voxels. Scalings of $(2,4,8)\times$ are also provided, where relevant. Each scaling effectively shortens each dimension to half of its previous scaling. Scaling $1\times$ thus has 8 times the number of voxels relative to scaling $2\times$.

Notice that a full simulation of the $1\times$ scaled tomogram was not possible in Novi-sim, due to the program crashing. This might either be a software or hardware limitation.

\begin{table}[]
\caption{Tomogram scalings and resolutions.}
\label{tab:tomogram_scales}
\centering
\begin{tabular}{ccc}
\toprule
Scale & Resolution $[\mu$m] & Shape \\ \midrule
1x & 1.875 & (3272, 3456, 3456) \\
2x & 3.75  & (1636, 1728, 1728) \\
4x & 7.5   & (818, 864, 864)    \\
8x & 15    & (409, 432, 432)    \\ \bottomrule
\end{tabular}
\end{table}

% TODO (Alle): Jeg vil mene at dette skal flyttes til Github, og/eller en
% README/HOWTO på ERDA hvor data også er (bliver?) tilgængelig.  det er nemlig
% kun relevant at forstå hvis man samtidigt også vil kigge på koden.
% ...
% Would also like to add something about what the individual integers in the
% ground truth represent, and how this numerical structure is decided (basically
% already described in code comment in masks/make_models.py)
