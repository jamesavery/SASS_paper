\section{Background}
\label{sec:background}

%TODO: Rephrase
% In this section, we will briefly go through the physical composition of our samples, and how the
% data is acquired. Then we will look at the noise sources typical for this type of data, and show
% the effects that noise has on regions around the implant and biological tissue.

\subsection{Background for the medical experiment} Installation of a dental implant initiates the
regional acceleratory phenomenon (RAP), which implies acceleration of the different healing stages.
RAP begins a few days after implant installation, peaks at 1-2 months, and subsides after 6-24
months~\cite{frost1989}. In cortical bone, the non-vital mineralized tissue initially needs to be
resorbed prior to bone formation. In the cancellous compartment, the implant installation mainly
results in damage of marrow spaces with resulting local bleeding and coagulum formation. The
coagulum gradually resorbs, collagen is laid down and replaced by osteoid, and eventually --- if
sufficient blood supply is present --- woven immature bone develops, and sequentially
osseointegration is initiated~\cite{frost1989}. After 6-12 weeks of healing most of the woven bone
is mineralized and bone marrow containing blood vessels, adipocytes, and mesenchymal cells can be
observed surrounding the trabeculae in the mineralized bone~\cite{Berglundh2003, Abrahamsson2004}.
A cement line, thickness  0.2-5µm, will be deposited directly on the implant surface during
continuous bone formation. The biological fixation of the implant initiates only a few days after
implant installation, where the osteoblasts begin to deposit collagen matrix on the cement line.
This early deposition of calcified matrix followed by the arrangement of woven bone and later mature
cancellous bone develops in a 3D manner delimiting the marrow space~\cite{Franchi2004}.

\subsection{Physical samples}

The experiment evaluated four methods for stimulating maxillary bone regeneration.  5 critical size
defects were introduced to 7 goats. Four defects were used to asses bone regeneration methods, and
one was a control sample.  Peri-implant vertical bone augmentation was performed using autologous
bone and two different calcium phosphate bone substitutes. The bone specimens were evaluated
undecalcified. The specimen preparation was performed at the Department of Biomaterials at
Gothenburg University, Sweden. The specimens were initially fixated in 4\% paraformaldehyde.
Dehydration of the specimens was performed in increasing concentrations of ethanol to eliminate fat
and water content. Furthermore, specimens were infiltrated with methylmethacrylate (MMA) and
embedded in molds 12 mm in diameter and 20 mm in height~\citep{NELDAM2015682}. They were scanned at
the ESRF in Grenoble, France. The advantage of using MMA is greater tissue penetration than
water-soluble methacrylates.  This is an advantage when preparing larger specimens such as bone
biopsies containing dental implants. Furthermore, the histological quality of bone sections is
generally higher for MMA embedded specimens compared to water-soluble
methacrylates~\cite{erben1997}. Additionally, tissue shrinkage is less than 2\% when using MMA
embedded bone and cartilage specimens~\cite{ferguson1999}.

Physical samples were prepared for SR$\mu$CT scanning by cutting out portions from the larger
cylindrical biopsies. Within these samples, we find the titanium dental implant (Astra Tech
OsseoSpeed, ST Molndal, Sweden).  It is 3.5mm in diameter and 8mm long. Along its length the lower
5.5mm has larger threads and is attached to recipient bone. The upper 2.5mm has smaller threads and
is where newly formed bone is to be assessed. Surrounding the bone and implant contact-region are
cavities containing resin, air, blood vessels and other fibrous tissue.

\begin{figure}
\centering
\includegraphics[width=0.96\columnwidth]{770c_pag-full-xy-1x.png}
\includegraphics[width=0.96\columnwidth]{770c_pag-full-yz-1x.png}
\includegraphics[width=0.96\columnwidth]{770c_pag-full-xz-1x.png}
\caption{Cut sample seen as cross sections in XY, YZ and XZ-planes respectively. A voxel has a size
of 1.875$\mu$m. Red voxels are numerically low, while blue voxels are high.}
\label{fig:3viewsample}
\end{figure}

% TODO: description of color depends on final colormap -- still better than lighter/dark
A cut sample is shown in three different cross sectional views in \Cref{fig:3viewsample}. Each
material has a unique density and thus absorption. The titanium implant shown in blue has a higher
absorption level than bone. Bone material shown in light orange has higher absorption than its
surrounding dark orange colored regions containing blood vessels tissue, air and resin.

\subsubsection{Data acquisition}

It can be difficult to study and evaluate the bone structure and blood network without destroying or
manipulating the sample. X-ray computed tomography is a widely used tool for non-intrusive medical
imaging. By exposing a subject to X-rays, we can map the linear attenuation coefficient of the
passing rays. Each ray is attenuated relatively to the density and composition of the material it
passes.  By rotating either the scanner or the sample we can get a full 3D image representation of
the inner structure of the sample. Each volumetric pixel (voxel) then represents the X-ray
attenuation at its spatial position. In this way, X-rays can reliably be used to internally characterise
samples in a non-intrusive and non-destructive manner. Medical CT-scans can provide spatial
resolutions on the order of submillimetre scale \citep{medicalct}. The more modern micro computed
tomography ($\mu$CT) can provide much higher spatial resolution on the micrometre scale
\citep{srexptime}.

This work focuses on a data set acquired by Synchrotron Radiation micro-CT (SR$\mu$CT). For this
imaging technique, electrons are accelerated to ultra-relativistic speeds in trajectories directed
by strong magnetic fields. The resulting X-ray beam provides a high photon flux allowing for very
short exposure times \citep{srexptime}. This can help counter Poisson noise from suboptimal photon
count \citep{srnoise}. Contrary to both CT and $\mu$CT, this approach requires a large particle
accelerator, and is not standard medical or laboratory equipment. However, SR$\mu$CT  offers an even
better spatial resolution of up to 0.1 $\mu$m, and much higher image quality due to fewer distortive
X-ray effects. The resulting beams are high in brilliance and collimation, which gives a very clear
signal. Artifacts from beam-hardening are minimized due to synchrotron radiation X-rays being
characeterised by their practically mono-energetic spectrum.

The tomograms prestened here have been acquired at the ID19 beam line at the European Synchrotron
Radiation Facility (ESRF) in Grenoble, France. They were reconstructed\citep{sporring} at the ID19
beamline. A standard filtered back-projection algorithm was applied via the ESRF in-house developed
software PyHST~\citep{NELDAM2015682}\citep{pyhst}. PyHST was applied to improve reconstruction
quality, hence reducing ring artefacts, and to reduce the required data volume if
necessary~\cite{MIRONE201441}. The tomograms were acquired at 50 KeV.

\subsubsection{Image data}

The physical field-of-view of a single image sample is about 6.5mm in each direction. Each sample
contains voxels with a spatial resolution of 1.875$\mu$m. The samples are scanned in chunks of 4-6
sub-volumes through the height of the implant, depending on the initial size of a sample.  For
computational purposes, the pixel size has been cropped to be divisible by a power $2^N$ or more
specifically $2^5=32$ was used in our case. This gives a size of $(3456,3456,846)$ pixels per
sub-volume, where the last axis gets stacked for a full volume. This gives a single image containing
4 subvolumes the dimensions $(3480,3480,3384)$ pixels in total. Before being stacked, we need to
volume match any overlapping regions. The overlaps are only shifted in the last dimension.  This
made it possible to find the best overlap by simply minimizing the pixel-wise Eucledian distance
between the volumes. The depth of the shifted volumes decreases from 846 to 810.

In image \Cref{fig:3viewsample} (YZ- and XZ-planes) we clearly see dark edges around the border of
the volume matched subvolumes at their overlaps. This creates an offset which is very prominent
within the implant, but the large relative contrast due to its high density, makes this easier to
ignore.  Even worse is the contribution of misrepresented voxels in the transitional regions where
bone and tissue contains visible jagged edges.


%%% Local Variables:
%%% mode: latex
%%% TeX-master: "main"
%%% End:
