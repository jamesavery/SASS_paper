Synchrotron Radiation micro-CT (SRµCT) produces 3D images at large spatial
resolution and high signal-to-noise ratio, but noise and
artifacts~\ref{adv_in_srmicroct} are still not eliminated in the reconstructed
image. This becomes particularly evident when attempting to obtain accurate
tissue classification near high-contrast interfaces, such as those introduced
by metal implants.

We present a computational method that discovers voxel intensity corruption as
a function of space and produces continuous probabilistic models of material
classification functions. Using the derived models, we can accurately classify
tissue throughout the full image, even at high-contrast transition interfaces.
We apply the method to solve the notoriously difficult problem of accurately
classifying soft biological tissue in contact with a dense titanium implant.

This new tissue classification method is used to evaluate bone-to-implant
contact (BIC) in micrometer-resolution SRµCT images. In a previous study, we
were unable to obtain accurate results for BIC, due to difficulties in
accurately classifying the tissue types near the titanium implant surface. In
the present work, we overcome the corruptive effects and obtain accurate tissue
classification close to the implant surface.
%TODO: Issue #1. MANGLER TAL HER. Disse skal beregnes.
The method is open-source, implemented in C++ and Python, and targets
out-of-core execution on multi-core CPU and multi-GPU systems.

%%% Local Variables:
%%% mode: latex
%%% TeX-master: "main"
%%% End:
