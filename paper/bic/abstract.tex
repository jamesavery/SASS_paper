Synchrotron Radiation micro-CT (SR$\mu$CT) produces images at
extremely high fidelity.  However, while distortive X-ray effects such
as beam-hardening are minimized due to highly brilliant monochromatic
beams, they are not eliminated. In particular, obtaining accurate
tissue classification is a challenge near high-contrast interfaces
such as metal implants.

We present a computational method that discovers the image distortion as a function of space,
and produces continuous probabilistic models of material classification functions.
Using the derived models, we are able to accurately classify tissue throughout the full
image, even at high-contrast transition interfaces.
% 
We apply the method to solve the notoriously difficult problem of accurately classifying
biological tissue in contact with a titanium implant. 
  
%{\bf The data:}
The new tissue classification method was used to evaluate
bone-to-implant contact (BIC) in micrometer-resolution SR$\mu$CT images.
In a previous study, we were unable to obtain accurate
results for BIC, due to difficulties in accurately classifying the
tissue types near the titanium implant surface. In the present work,
we fully reverse the distortive effects and obtain accurate tissue classification
all the way to the implant surface.

The method is implemented in C++ and Python, and is parallelized for GPU and multi-core CPU.
To deal with the very large image sizes arising from SR$\mu$CT, exceeding system
memory on even large workstations, the algorithms are designed to run
out-of-core on multi-resolution representations of the tomograms.

The source code is available as open source at http://github.com/jamesavery/SASS/.



%%% Local Variables:
%%% mode: latex
%%% TeX-master: "main"
%%% End:
