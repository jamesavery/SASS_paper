
\section{Introduction}
\label{sec:intro}


\subsection{Image data}

The most common method for analysing bone samples is to extract histologies and examine their two-dimensional structure with light microscopy.
This method has several drawbacks. First and foremost, it is destructive: in addition to the obvious issue that the histology must be cut
from the full sample, the sawing process can contaminate soft tissue with bone dust, or leave surface scratches that complicates automatic image analysis.
Secondly, histology by its nature only gives a two-dimensional slice of the full three-dimensional picture. The most important biological structures
are inherently three-dimensional, and limiting analysis to 2D severely restricts the types of questions we can answer. 

Synchrotron Radiation micro-tomography (SR$\mu$CT) offers a non-destructive high-quality alternative to histology for detailed analysis of bone biopsies.
Lauridsen et al.~\cite{torsten2018} quantified the uncertainty of 2D histology for four common bone analyses, and found that the choice of sampling
plane for histological analysis incurred a significant uncertainty in the results, whereas the full volumetric analysis of SR$\mu$CT tomograms did not.

Further, the high brilliance and collimation of synchrotron radiation yields particularly faithful 3D images, as the common distortive X-ray effects
in hospital-grade setups such as beam hardening and projection artefacts are minimized. The high fidelity
makes SR$\mu$CT attractive for conducting advanced medical image analyses with trustworthy results.

However, while image distortion effects are much reduced compared to mainstream X-ray tomography, they are not eliminated, and numerical
analysis and computations on the images must still be conducted carefully. Boundary
effects near sample surfaces, ring artefacts, and especially distortion near high-contrast transitions, make accurate tissue classification
difficult in regions where this distortion is significant. Neldam et al.~\cite{sporring} found that, while they could accurately classify
bone tissue in the middle regions of the tomograms, they were not able to obtain good bone-to-implant contact evaluation (BIC), as evidenced
by poor correlation with histological analysis of the same samples.

\james{Her følger paragraf fra Else om, hvorfor BIC er vigtigt for at evaluere regenereret knogle.}

The present work presents a computational method which automatically discovers probabilistic models for the distortions incurred by the physical effects in
high-resolution X-ray CT such as SR$\mu$CT in order to reverse them and produce accurate tissue classification even in regions where these effects are significant.
It exploits two properties, which are needed to hold for the method to work:
i) very high resolution is used to build statistical models as functions of space, and ii) that the effects to be coutnered vary continuously
over space, so that we can track how voxel frequency distributions are distorted through space. We apply the method to the same dataset
of micrometer-resulution SR$\mu$CT bone tomograms studied in \cite{torsten2018} and in \cite{sporring}, to achieve faithful tissue classification
all the way to the titanium implant surface.



% \begin{itemize}
%  \item Vi undersøger:
%      \begin{itemize}
%      	\item Overordnet: Imlantantkontakt i knogler
%      	\item Vil gerne kunne analysere og evaluere kvaliteten af "growth" omkring implantatet for at kunne vurdere den underlæggende sundhed og success af operationen
%      	\item Vil gerne kunne afkorte tiden fra operation til analyse til evaluering, samtidigt med at kvaliteten af data er stabil til at der er højt evidensgrundlag
%      	\item kunne sige noget om knoglekvaliteten og kompositionen
%      	\item kontakt til implantat: dækket af knogle, blod (forbundet til blodnetværk)
%      \end{itemize}
%  \item Til det skal løses:
%      \begin{itemize}
% 	\item SR$\mu$CT enables this approach due to the quality and resolution of data
%      	\item Segmentation is hard and time-consuming to do on this type of data.
%      	\item contains noise that is hard to avoid from data acquisition
%      	\item often segmentation is done manually, so being able to automate this data is of great value for the field.
%      	\item Want to avoid using deep learning methods, but keep full determinism and transparency 
%      	\item Simpler mathematical methods can be implemented efficiently and consistently
%      \end{itemize}
%  \item Hvordan vil vi bære os ad med at løse problemerne?
%      \begin{itemize}
%      	\item Det er en forudsætning at vi tydeligt kan observere hvad der sker i bone-to-implant interfacet
%      	\item Den utrolig høje opløsning og kvaliteten af billeder fra SR-micro-CT giver adgang til analyse og segmentering med høj nøjagtighed.
% %     	\item Ved brug 
%      \end{itemize}
% \end{itemize}




% What is it we want to investigate?
%...

% What are the challenges?
%...

% How will we solve these challenges?
%...

%%% Local Variables:
%%% mode: latex
%%% TeX-master: "main"
%%% End:
