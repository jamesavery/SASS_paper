\section{Introduction}
\label{sec:intro}


\subsection{Image data}

Bone samples are most commonly analysed by extracting histologies and examining their
two-dimensional structure with light microscopy. This method has several drawbacks. First and
foremost, it is destructive: In addition to the obvious issue that the histology must be cut
from the full sample, the sawing process can contaminate soft tissue with bone dust, or leave
surface scratches that complicate automatic image analysis. Secondly, histology by its nature
only gives a two-dimensional slice of the full three-dimensional picture. Most important
biological structures are inherently three-dimensional, and limiting analysis to 2D severely
restricts the types of questions we can answer.

Synchrotron Radiation micro-tomography (SR$\mu$CT) offers a non-destructive high-quality
alternative to histology for detailed analysis of bone biopsies. \cite{torsten2018}
quantified the uncertainty of 2D histology for four common bone analyses, and found that the
choice of sampling plane for histological analysis incurred a significant uncertainty in the
results, whereas the full volumetric analysis of SR$\mu$CT tomograms did not.

The high brilliance and collimation of synchrotron radiation yields particularly
faithful 3D images, as common distortive X-ray effects seen in hospital-grade setups such
as beam hardening and projection artefacts are minimized. The high fidelity makes SR$\mu$CT
attractive for conducting advanced medical image analyses with trustworthy results.

However, while image distortion effects are much reduced compared to laboratory X-ray tomography,
they are not eliminated, and numerical analysis and computations on the images must still be
conducted carefully. Boundary effects near sample surfaces, ring artefacts from sensor faults, and especially
distortion near high-contrast transitions, make accurate tissue classification difficult in
regions where this distortion is significant. \cite{sporring} found that, while
they could accurately classify bone tissue in the middle regions of the tomograms, they were
not able to obtain good bone-to-implant contact evaluation (BIC), as evidenced by poor correlation
with histological analysis of the same samples.

The present work presents a fully automatic computational method which discovers probabilistic
models for the distortions incurred by the physical effects in high-resolution X-ray CT such
as SR$\mu$CT in order to reverse them and produce accurate tissue classification even in regions
where these effects are significant. It exploits two properties, which are needed to hold for
the method to work: i) very high resolution is used to build statistical models as functions
of space, and ii) the effects to be countered must vary continuously over space, so that we can
track how voxel frequency distributions are distorted throughout the image.
We apply the method to the
same dataset of micrometer-resolution SR$\mu$CT bone tomograms studied in \cite{torsten2018}
and in \cite{sporring}, to achieve faithful tissue classification all the way to the titanium
implant surface.

Our goal is to obtain good conditional probability distributions $P(m|v,\xx)$
that model the likelihood of a voxel having material type $m$ as a function
both on its value $v$ and its position $\xx$ in the tomogram. We want to make sure that these distribution
functions vary smoothly across space, to ensure that we can identify the materials correctly across the entire
image: even though the frequency distributions look completely different
close to the titanium implant compared to the middle region or sample surface,
we can track the unbroken, smooth deformation to assign a global material
identity.

The aims of the current work is:
\begin{enumerate}
\item To design a fully automatic \textbf{spatially aware segmentation algorithm} that
  improves segmentation quality of tissues in bone-SR$\mu$CT in all regions, including near high-contrast interfaces.
\item To implement this method efficiently in open-source software for GPU and multi-core CPU using out-of-core techniques, to
  facilitate analysis of 3D SR$\mu$CT images that exceed system memory.
\item To use the new method to evaluate bone-to-implant contact closer to implant surfaces than previously feasible. 
\end{enumerate}




%%% Local Variables:
%%% mode: latex
%%% TeX-master: "main"
%%% End:
