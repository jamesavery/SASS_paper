
\section{Introduction}
\label{sec:intro}

\carl{Maybe a subsection describing how the old paper falls short close to the implant? I'm sure
it'll be in the introduction as well, but shouldn't it also be explained in detail here? Maybe
just a more detailed walkthrough of how they did it, and why it didn't work?}

\aleksandar{Rykket fra motivation her til, efter kort diskussion med Carl.}

% \begin{itemize}
%  \item Vi undersøger:
%      \begin{itemize}
%      	\item Overordnet: Imlantantkontakt i knogler
%      	\item Vil gerne kunne analysere og evaluere kvaliteten af "growth" omkring implantatet for at kunne vurdere den underlæggende sundhed og success af operationen
%      	\item Vil gerne kunne afkorte tiden fra operation til analyse til evaluering, samtidigt med at kvaliteten af data er stabil til at der er højt evidensgrundlag
%      	\item kunne sige noget om knoglekvaliteten og kompositionen
%      	\item kontakt til implantat: dækket af knogle, blod (forbundet til blodnetværk)
%      \end{itemize}
%  \item Til det skal løses:
%      \begin{itemize}
% 	\item SR$\mu$CT enables this approach due to the quality and resolution of data
%      	\item Segmentation is hard and time-consuming to do on this type of data.
%      	\item contains noise that is hard to avoid from data acquisition
%      	\item often segmentation is done manually, so being able to automate this data is of great value for the field.
%      	\item Want to avoid using deep learning methods, but keep full determinism and transparency 
%      	\item Simpler mathematical methods can be implemented efficiently and consistently
%      \end{itemize}
%  \item Hvordan vil vi bære os ad med at løse problemerne?
%      \begin{itemize}
%      	\item Det er en forudsætning at vi tydeligt kan observere hvad der sker i bone-to-implant interfacet
%      	\item Den utrolig høje opløsning og kvaliteten af billeder fra SR-micro-CT giver adgang til analyse og segmentering med høj nøjagtighed.
% %     	\item Ved brug 
%      \end{itemize}
% \end{itemize}




% What is it we want to investigate?
%...

% What are the challenges?
%...

% How will we solve these challenges?
%...

%%% Local Variables:
%%% mode: latex
%%% TeX-master: "main"
%%% End:
