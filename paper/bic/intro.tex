\section{Introduction}
\label{sec:intro}

Bone samples are most commonly analyzed by extracting histologies and examining
their two-dimensional structure with light microscopy. This method has several
drawbacks. First and foremost, it is destructive: In addition to the obvious
issue that the histology must be cut from the full sample, the sawing process
can contaminate soft tissue with bone dust, or leave surface scratches that
complicate automatic image analysis. Secondly, histology by its nature only
gives a two-dimensional slice of the full three-dimensional picture - most
important biological structures are inherently three-dimensional, and limiting
analysis to 2D severely restricts the types of questions we can answer.

Synchrotron Radiation micro-tomography (SRµCT) offers a non-destructive
high-quality alternative to histology for detailed analysis of bone biopsies.
\cite{torsten2018} quantified the uncertainty of 2D histology for four common
bone analyses, and found that the choice of sampling plane for histological
analysis incurred a significant uncertainty in the results, whereas the full
volumetric analysis of SRµCT tomograms did not.

The high brilliance and collimation of synchrotron radiation yields
particularly faithful 3D images, as common distortive X-ray effects seen in
hospital-grade setups such as beam hardening and projection artifacts are
minimized. The high fidelity makes SRµCT attractive for conducting advanced
medical image analyses with trustworthy results.

However, while image distortion effects are much reduced compared to laboratory
X-ray tomography, they are not eliminated, and numerical analysis and
computations on the images must still be conducted carefully. Boundary effects
near sample surfaces, ring artifacts from sensor faults, and especially
distortion near high-contrast transitions, make accurate tissue classification
difficult in regions where this distortion is significant. \cite{sporring}
found that, while they could accurately classify bone tissue in the middle
regions of the tomograms, they were not able to obtain good bone-to-implant
contact evaluation (BIC), as evidenced by poor correlation with histological
analysis of the same samples.

% TODO: PLACEHOLDER FOR LITERATURE REVIEW - why are the current methods not good enough?
% 
% Include weaknesses and limitations of:
%  - Morphology-based-article
%  - software approaches: Pore3D, Avizo, ITK-Snap, ..., no one-fits-all program
%     - also very expensive, time-consuming and manual work already for much smaller volumes
%  - ML-based-article with U-net
%  - include article G with their systematic literature review also

The present work presents a fully automatic computational method which
discovers probabilistic models for the distortions incurred by the physical
effects in high-resolution X-ray CT such as SRµCT to reverse them and
produce accurate tissue classification even in regions where these effects are
significant. It exploits two properties, which are needed to hold for the
method to work: i) very high resolution is used to build statistical models as
functions of space, and ii) the effects to be countered must vary continuously
over space so that we can track how voxel frequency distributions are distorted
throughout the image. We apply the method to the same dataset of
micrometer-resolution SRµCT bone tomograms studied in \cite{torsten2018}
and in \cite{sporring}, to achieve faithful tissue classification close to the
titanium implant surface. The implementation is open-source, targets multi-CPU
and multi-GPU systems, and is available at
\href{https://github.com/jamesavery/maxibone}
{https://github.com/jamesavery/maxibone}.

%%% Local Variables:
%%% mode: latex
%%% TeX-master: "main"
%%% End:
