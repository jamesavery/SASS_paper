\section{Introduction}
\label{sec:intro}


\subsection{Image data}

The most common method for analysing bone samples is to extract histologies and examine their
two-dimensional structure with light microscopy. This method has several drawbacks. First and
foremost, it is destructive: In addition to the obvious issue that the histology must be cut
from the full sample, the sawing process can contaminate soft tissue with bone dust, or leave
surface scratches that complicates automatic image analysis. Secondly, histology by its nature
only gives a two-dimensional slice of the full three-dimensional picture. The most important
biological structures are inherently three-dimensional, and limiting analysis to 2D severely
restricts the types of questions we can answer.

Synchrotron Radiation micro-tomography (SR$\mu$CT) offers a non-destructive high-quality
alternative to histology for detailed analysis of bone biopsies. \cite{torsten2018}
quantified the uncertainty of 2D histology for four common bone analyses, and found that the
choice of sampling plane for histological analysis incurred a significant uncertainty in the
results, whereas the full volumetric analysis of SR$\mu$CT tomograms did not.

Further, the high brilliance and collimation of synchrotron radiation yields particularly
faithful 3D images, as the common distortive X-ray effects in hospital-grade setups such
as beam hardening and projection artefacts are minimized. The high fidelity makes SR$\mu$CT
attractive for conducting advanced medical image analyses with trustworthy results.

However, while image distortion effects are much reduced compared to mainstream X-ray tomography,
they are not eliminated, and numerical analysis and computations on the images must still be
conducted carefully. Boundary effects near sample surfaces, ring artefacts, and especially
distortion near high-contrast transitions, make accurate tissue classification difficult in
regions where this distortion is significant. \cite{sporring} found that, while
they could accurately classify bone tissue in the middle regions of the tomograms, they were
not able to obtain good bone-to-implant contact evaluation (BIC), as evidenced by poor correlation
with histological analysis of the same samples.

Installation of a dental implant initiates the regional acceleratory phenomenon (RAP), which
implies acceleration of the different healing stages. RAP begins a few days after implant
installation, peaks at 1-2 months, and subsides after 6-24 months~\cite{frost1989}. In cortical bone,
the non-vital mineralized tissue initially needs to be resorbed prior to bone formation. In the
cancellous compartment, the implant installation, mainly results in damage of marrow spaces with
resulting local bleeding and coagulum formation. The coagulum gradually resorbs, collagen is laid
down and replaced by osteoid, and eventually—if sufficient blood supply is evident—woven immature
bone  develops, and sequentially osseointegration is initiated~\cite{frost1989}. After 6-12 weeks of
healing most of the woven bone is mineralized and bone marrow containing blood vessels, adipocytes,
and mesenchymal cells can be observed surrounding the trabeculae in the mineralized bone~\cite{Berglundh2003, Abrahamsson2004}.
A cement line, thickness  0.2-5µm, will be deposited
directly on the implant surface during continuous bone formation. The biological fixation of the
implant initiates only a few days after implant installation, where the osteoblasts begin to deposit
collagen matrix on the cement line. This early deposition of calcified matrix followed by the
arrangement of woven bone and later mature cancellous bone develops in a 3D manner delimiting the
marrow space~\cite{Franchi2004}.

The present work presents a computational method which automatically discovers probabilistic
models for the distortions incurred by the physical effects in high-resolution X-ray CT such
as SR$\mu$CT in order to reverse them and produce accurate tissue classification even in regions
where these effects are significant. It exploits two properties, which are needed to hold for
the method to work: i) Very high resolution is used to build statistical models as functions
of space, and ii) that the effects to be countered vary continuously over space, so that we can
track how voxel frequency distributions are distorted through space. We apply the method to the
same dataset of micrometer-resolution SR$\mu$CT bone tomograms studied in \cite{torsten2018}
and in \cite{sporring}, to achieve faithful tissue classification all the way to the titanium
implant surface.


%%% Local Variables:
%%% mode: latex
%%% TeX-master: "main"
%%% End:
