\section{Method}
\label{sec:method}

% Spatial correlation 2d histograms (den nemme, 1 2d hist)
In this section, we will describe the method for exploiting spatial correlation
based on the 2D histograms of the tomographies. We will start by motivating the
problem and then give an overview of how we solve the problem, with a final
detailed walk-through of each step in the overall workflow.

% TODO  preprocessing
%As the scans slightly overlap, we first compute the overlaps between volumes,
%by shifting along the $Z$-axis until the square 3D image differences are
%minimized over the overlap volume, producing the best volume match. This
%allows us to combine the sub-volumes into a single coherent 3D image of the
%full sample. The images are represented in a custom hierarchical image format
%to facilitate fast multiresolution analysis. The full image has resolution
%$3456 \times 3456 \times 3360$ (ensured to be divisible by $32=2^5$), and has
%5 coarse layers at resolution divided by 2, 4, 8, 16, and 32.

\subsection{1-dimensional histograms}
%diskuter overlappende materialer
Looking at a 1D-histogram of the voxel value in a tomography, as shown
in~\Cref{fig:1d-hist}, we can distinguish different distributions, but we see
that there are large overlaps. This leads to global thresholding being
infeasible, at least for the distributions in the lower half of the histogram.
This happens because the voxel value is not globally defined,
as~\Cref{sec:physics} explains, which illustrates how the different materials
cover ranges of values that blend together in the histogram.

\begin{figure}
    \centering
    \includegraphics[width=\linewidth]{1d_hist.pdf}
    \caption{1-dimensional histogram of the voxel values of a tomography. The
    blue line is the air peak, the red line is the soft tissue peak, the orange
    line is the bone peak, and the green line is the implant peak.}
    \label{fig:1d-hist}
\end{figure}

\subsection{Preprocessing}\label{sec:preprocess}
To reduce the complexity of the segmentation process, we preprocess
the data by removing redundant information.
% Since the values of the implant material are well-separated from the rest of
% the sample, obtaining the rough implant mask through global thresholding is
% trivial, as the orange line in~\Cref{fig:1d-hist} shows. This mask is then
% refined, along with the internal gaps being closed, giving us the implant
% mask.
We first compute a coarse bone region using a crude segmentation. Through
direct geometric analysis of the implant, we determine the sample coordinate
system, with the origin on the back-plane of the sawed-through implants and
principal axes coordinate system vectors $\mathbf{u}$ pointing up towards the
implant top, $\mathbf{v}$ pointing forward away from the back-plane, and
$\mathbf{w}$ point right, parallel to the back-plane. An automatic
wave-analysis of the threads computationally determines the macro-threaded
recipient bone region, and the micro-threaded de-novo regenerated bone region.
We then perform the spatially-aware segmentation analysis restricted to the
bone regions to not expend effort on the parts of the segmentation
problem that can be solved with simpler methods. Note that the process is fully
automatic, and does not require human intervention.

%TODO: Måske medtage disse detaljer, men er for træt til at renskrive.
%, we are also able to remove the air making up half of the sample cylinder.
%This is depicted in~\Cref{fig:3viewsample}, above the implant and to the left
%of the implant in the XY and YZ cross sections respectively. Finally, the
%implant mask can show us the orientation of the implant. This squashes the
%blue peak from the histogram, leaving the two peaks marked by the red and
%green lines as being the most defined.

% As we are only interested in the voxels inside the bone, and we know that
% bone is the most prominent material left, we perform another coarse
% segmentation. This is done by thresholding between the two large
% distributions, choosing everything to the right of this threshold as the bone
% mask. These voxels are morphologically closed, to also consume the internal
% gaps in the bone region, which makes up the soft tissue inside the bone. The
% implant and bone region masks are applied throughout subsequent steps to
% filter out noise in the tomography.

\subsection{Exploiting spatial information}
% Show the correlation of x,y,z,r to histograms 2d
Our goal is to discover how the value distributions for the different materials
-- bone, blood vessels, etc. -- change as a function of space. In other words,
we wish to uncover information about the conditional probabilities
$\Pof{m}{v,\xx}$: that a voxel with value $v$ and position $\xx$ represents
material $m$. We cannot compute this directly using only the image, as only one
voxel occupies position $\xx$.

However, if we fix one axis $x$, $y$, or $z$, the image {\em does} contain
millions of voxels with that fixed value, enough to make good statistical
models for e.g.~the conditional probabilities $\Pof{m}{v,x}$, $\Pof{m}{v,y}$,
and $\Pof{m}{v,z}$. To this end, we compute 2D histograms that count voxel
frequencies both conditioned on value $v$ and coordinate value along either
$x$, $y$, or $z$.

Figure \ref{fig:2dhists}(a) shows the histogram for our model tomogram (implant
excluded) as a function of the $y$ coordinate: each row in the image is a
histogram for a fixed value of $y$. Figure \ref{fig:3viewsample}(a) helps us
see what happens: For $y<2400\text{µm}$, there is only air, then we reach a
thin layer of resin, after which we enter the region where we find bone, soft
tissue, resin, and air. The air voxels are brightened as we approach the
implant, shifting the peaks smoothly rightwards. A better coordinate system for
capturing both the distortive edge effects and the glow from the implant is to
use the cylindrical coordinates from the tomography center line. Figure
\ref{fig:2dhists}(b) shows a similar 2D histogram for the radial cylinder
coordinate $r=\sqrt{\left( x-c_x(z)\right)^2+\left( y-c_y(z)\right)^2}$. From
this plot, we see two prominent distributions that change along the $r$ axis.
The radius correlates with distance to the sample surface (capturing edge
effects), and for medium $r$, it is a good proxy for distance to the implant,
and we see a brightening with smaller $r$, and a darkening and broadening of
the distributions for large $r$. Each {\em view} provides us with additional
information about how voxel values are distorted throughout space: analogous to
casting shadows along different axes to obtain more information about a 3D
object. We can either use Bayesian statistics to combine information from
multiple axes or construct a spatial grouping that is particularly suited to
capture the effects of the distortive effects that we want to counter. The next
section describes the latter.

\begin{figure}
    \centering
    %\includegraphics[width=.49\linewidth]{zb-bone_region3.png}
    %\includegraphics[width=.49\linewidth]{yb-bone_region3.png}
    %\includegraphics[width=.49\linewidth]{xb-bone_region3.png}
    %\includegraphics[width=\linewidth]{rb-bone_region3.png}
    \begin{tabular}{cc}
      \!\!\!\!\!\!(a) \begin{tabular}{c}\includegraphics[width=0.7\columnwidth]{yb-full3.png}\end{tabular}\\
      \!\!\!\!\!\!(b) \begin{tabular}{c}\includegraphics[width=0.7\columnwidth]{rb-full3.png}\end{tabular}\\
      \!\!\!\!\!\!(c) \begin{tabular}{c}\includegraphics[width=0.7\columnwidth]{fb-edt-full2.png}\end{tabular}\\
%      \includegraphics[width=\linewidth]{rb-bone_region3.png}\\
    \end{tabular}
    \caption{Examples of 2D histograms for the full tomogram: (a) along the
    $y$-axis, (b) as a function of distance $r$ to the center, and (c) as a
    function of distance $d$ to the implant. The abscissa of the 2D histograms
    is the voxel value and the ordinate is the value of $y$ resp.~$r$ and $d$.
    Notice that the $r$-2D histogram well separates the materials for large and
    intermediate values of $r$, but breaks down for small $r$. However, it
    clearly shows a brightening trend for small $r$.
    }

    \label{fig:2dhists}
\end{figure}

\subsection{Field histograms}
%Show that the fields can perfectly separate
%discuss all the way to implant contact
In the present work, the main distortion that we want to invert is the
brightening of voxels near the high-contrast interface between titanium implant
and biological tissue, to accurately determine tissue-implant contact. To this
end, we can group the voxels according to their distance from the implant using
the {\em Euclidean Distance Transform} (EDT). \Cref{fig:2dhists}(c) shows the
corresponding 2D histogram, which shows a darkening effect for large distances
(near the sample surface) and brightening for small distances (near the implant
surface).

%EDT is good overall, but difficult close to the implant.
However, we can do better: As discussed in~\Cref{sec:physics}, the implant
produces a \textit{glowing} effect, which can be modeled physically by
diffusion. \Cref{fig:edt-vs-diffusion} shows how a voxel inside the troughs of
the threading receives brightening contributions from many sides, while a voxel
above the peak of the threading at the same distance from the implant receives
much less. To resolve the brightening of voxels very close to the implant, it
thus makes sense to use a diffusion field, as shown in \Cref{fig:field-slice}.

\begin{figure}
    \centering
    \includegraphics{glowing-crop}
    \caption{Visualization of glowing effect close to the implant surface,
        shown in the YZ plane. The two points $p_1$ and $p_2$, marked in red, have
        the same distance to the implant but receive a markedly different
        brightening effect. Diffusion is depicted in green, where we see multiple
        arrows contributing to the value of $p_1$. The dotted line depicts a
        constant radius from the tomogram center.}
    \label{fig:edt-vs-diffusion}
\end{figure}


\begin{figure}
    \includegraphics[width=\linewidth]{770c_pag-gauss-yz.png}
    \caption{A slice of the diffusion field in the YZ plane.}
    \label{fig:field-slice}
\end{figure}

By using a diffusion field to model the glow from the implant, we can describe
the brightening effect even very close to the implant, but the field is nearly
zero everywhere else. To construct a single field that separates voxels well
according to the major contributions of distortion, we construct a combined
EDT+diffusion field. We will see in the next section that this is enough to
produce a high-quality segmentation throughout the tomogram, both near the
implant surface and far away from it.

% Show that the distance to the implant in EDT will be the same in the grooves
% of the screw compared to the threads of the screw. This is "fixed" with
% diffusion, as the grooves will be brighter as it is surrounded by more
% implant.

%TODO: Adskil METODE, altsaa SASS, fra ANVENDELSE, altsaa BIC.
\subsection{Walk-through of the method}
This section will describe each step of our method in detail. Going from
tomography to a segmented tomography.

\subsubsection{Overview}
%Coarse steps, explain the idea
To reach the tissue-bone implant contact metric, we have the three coarse
steps: compute the fields, segmentation using the fields, and extraction of the
contact from the segmented tomography.
%
The steps of the field computation are:
\begin{enumerate}
    \item[1.] Compute the EDT and Diffusion fields to give each voxel spatial
        information about its relation to the implant.

    \item[2.] Compute frequency distributions of voxel values as functions of
        field values as 2D histograms.
\end{enumerate}
%
Then, to segment using the fields:
\begin{enumerate}
    \item[3.] Find the material ridges within the 2D histograms, using image
        processing techniques.

    \item[4.] From the ridges, compute the initial approximate frequency
        distributions of each material, which are then optimized to fit the 2D
        histograms. From the optimized distributions, we derive probability
        distributions for material classification. These distributions
        approximate the conditional probabilities $\Pof{m}{v,x}$ that a
        particular voxel belongs to material $m$ given that it has voxel value
        $v$ and field value $x$.

    \item[5.] Apply the probability distributions to the tomography, segmenting
        the voxels into the different materials.
\end{enumerate}
%
In the present paper, we use the improved segmentation to evaluate bone- and
blood-implant contact.
\begin{enumerate}
    \item[6.] From the segmented tomography, we compute the network of blood
        vessels as the largest connected component of the soft-tissue material.
        Bone is classified as the largest connected component of the bone
        material that is within a certain distance of the blood vessel network.

    \item[7.] Perform the implant contact analysis.
\end{enumerate}
%
These steps are also summarized in the flow chart in~\Cref{fig:flowchart}.

\begin{figure}
  \centering
  \includegraphics{steps}
    \caption{Flowchart depicting the steps of the method.}
    \label{fig:flowchart}
\end{figure}

\subsubsection{Segmentation}
The overall segmentation is computed in steps 1-5. For each step we will
describe the process, showing the algorithm where applicable, along with the
intermediate results.

\vspace{\baselineskip}
\noindent\textit{\textbf{Step 1: Field computations}}\\
Both fields are computed from the implant mask, as described
in~\Cref{sec:preprocess}. EDT is computed in parallel using W.~Silversmith's
implementation of Meijster's algorithm \cite{pypi-edt}, yielding a 3D image in
which every non-implant voxel is the Euclidean distance to the nearest implant
voxel.

For diffusion, rather than solving a full diffusion equation, we approximate it
using repeated convolutions of a 3D-Gaussian kernel, implemented by separating
into 1D convolutions, as seen in~\Cref{alg:diffusion}, and the XZ-plane
in~\Cref{fig:field-slice}.

\begin{algorithm}
    \caption{Diffusion approximation.}
    \label{alg:diffusion}
    \begin{algorithmic}
        \Function {Diffusion} {voxels$[n_z*n_y*n_x]$, $n_\text{repetitions}$, \newline \indent \indent kernel$[2*k+1]$}
            \State $S \gets [n_y * n_x, n_x, 1]$
            \State $N \gets [n_z, n_y, n_x]$
            \State $\text{buf}_0[:] \gets \text{voxels}[:]$
            \For {$\text{rep}$ \textbf{in} $0{:}n_\text{repetitions}$}
                \For {$\text{dim}$ \textbf{in} $0{:}3$}
                    \For {$z,y,x$ \textbf{in} $0{:}n_z,0{:}n_y,0{:}n_x$}
                        \State $X \gets [z,y,x]$
                        \State $i_\text{start} \gets - \min (k, X[\text{dim}])$
                        \State $i_\text{end} \gets \min (k, N[\text{dim}] - X[\text{dim}] - 1)$
                        \State $i_\text{global} \gets z*S[0] + y*S[1] + x*S[2]$
                        \For {$i$ \textbf{in} $i_\text{start}:i_\text{end}$}
                            \State $i_\text{offset} \gets i_\text{global} + i*S[\text{dim}]$
                            \State $\text{buf}_1[i_\text{global}] \gets \text{buf}_0[i_\text{offset}] * \text{kernel}[i+k]$
                        \EndFor
                    \EndFor
                    \State $\text{buf}_0[:] \gets \text{buf}_1[:]$
                \EndFor
            \EndFor
            \State \Return $\text{buf}_0$
        \EndFunction
    \end{algorithmic}
\end{algorithm}

%EDT + Diffusion
To obtain a good separation both close to the implant and far away, we combine
the two fields into a single one as shown in~\Cref{eq:field-comb}.

%\begin{algorithm}
%    \caption{Field combination}
%    \label{alg:field-comb}
%    \begin{algorithmic}
%        \Function {field\_combine} {$f_{edt}, f_{dif}$}
%            \State $r \gets f_{dif} - \frac{f_{edt}}{\max (f_{edt})}$
%            \State $r \gets r - \min (f_{dif})$
%            \State $r \gets \frac{r}{\max (f_{dif})}$
%
%            \State \Return $r$
%        \EndFunction
%    \end{algorithmic}
%\end{algorithm}

\begin{equation}
    \label{eq:field-comb}
    \begin{split}
        f_{\text{diffusion\_norm}} &= \frac{f_{\text{diffusion}} - \min (f_{\text{diffusion}})}{\max (f_{\text{diffusion}})} \\
        f_{\text{edt\_norm}} &= \frac{f_{\text{edt}} - \min (f_{\text{edt}})}{\max (f_{\text{edt}})} \\
        f_{\text{combined}} &= f_{\text{diffusion\_norm}} - f_{\text{edt\_norm}} \\
        f_{\text{combined\_norm}} &= \frac{f_{\text{combined}} + \|\min (f_{\text{combined}})\|}{\max (f_{\text{combined}})}
    \end{split}
\end{equation}

\vspace{\baselineskip}
\noindent\textbf{Step 2: 2D histograms} \\
From the combined fields we compute a 2D histogram with the field value on the
y-axis and the voxel value on the x-axis. The algorithm for the field histogram
can be seen in~\Cref{alg:field-hist} along with a plot of the resulting
histogram in~\Cref{fig:field-hist}. We see that the bone and soft tissue
separate neatly into two distinguishable distributions. The effect of the
diffusion field is to ``zoom in'' near the implant surface, where the diffusion
field changes rapidly.

\begin{algorithm}
    \caption{Field 2D histograms.}
    \label{alg:field-hist}
    \begin{algorithmic}
        \Function {Field\_hist} {$\text{voxels}[n_z,n_y,n_x]$, $\text{field}[n_z,n_y,n_x]$, $v_\text{bins}$, $v_
        \text{min}$, $v_\text{max}$, $f_\text{bins}$, $f_\text{min}$, $f_\text{max}$}
            \For {$z,y,x$ \textbf{in} $0{:}n_z,0{:}n_y,0{:}n_x$}
                \State $v \gets \text{voxels}[z,y,x]$
                \If {$v_\text{min} \leq v \leq v_\text{max}$}
                    \State $f \gets \text{field}[z,y,x]$
                    \If {$f_\text{min} \leq f \leq f_\text{max}$}
                        \State $v_i \gets (v_\text{bins} - 1) - \frac{v - v_\text{min}}{v_\text{max} - v_\text{min}}$
                        \State $f_i \gets (f_\text{bins} - 1) - \frac{f - f_\text{min}}{f_\text{max} - f_\text{min}}$
                        \State $h[f_i,v_i]{+}{+}$
                    \EndIf
                \EndIf
            \EndFor
            \State \Return $h$
        \EndFunction
    \end{algorithmic}
\end{algorithm}

\begin{figure}
    %\includegraphics[width=\linewidth]{fb-edt-bone_region3.png}
    \includegraphics[width=\linewidth]{fb-gauss+edt-bone_region3.pdf}
    \caption{
        2D field histogram. Note that there are two clearly separated
        ridges, which each represent the expectation value for a particular
        material as a function of the field value.
    }
    \label{fig:field-hist}
\end{figure}

\vspace{\baselineskip}
\noindent\textbf{Step 3: Identify the materials} \\
To decompose a 2D histogram into a sum of probability distributions that
represent the different materials present in the tomogram, we first identify
the {\em ridges}: i.e., the peaks in the 1D histograms that are persistent and
vary continuously with the field value. Thus we can expect each ridge to
represent the {\em expectation values} for a particular material as a function
of the field value, i.e., as a function of the voxel's position in space.
%
Any ridge-finding method will do: we compute it through the series of
image-processing operations shown in \Cref{alg:material}.
% The next step is to isolate the different materials spatially, which is done
% through the field histogram. For our first attempt, we applied Otsu's
% threshold to each row in the 2D histogram. While this provided good results
% for two well-defined distributions, it eventually began to fail. This was
% either because of a lack of samples, or because the histogram did not contain
% two distributions, which is an underlying assumption of Otsu thresholding.
% These two effects occurred at the edges (both closest and furthest from the
% implant), which would bleed into the later steps of the segmentation process.

% Rather than finding the threshold value in between the distributions, our
% next solution was to start finding the distributions by finding the ridges of
% the 2D-histogram. We do this by applying a series of image processing
% techniques, described in~\Cref{alg:material}. Each of the resulting contours
% becomes its own material, which is then labeled as material 1 or 2
% respectively for this particular sample. For samples that would contain more
% than two distributions, the same algorithm should apply, as long as the
% distributions do not overlap.

\begin{algorithm}
    \caption{2D histogram ridge-finding.}
    \label{alg:material}
    \begin{algorithmic}
        \Function {Ridges} {$h[f_\text{bins}$, $v_\text{bins}]$, $\sigma$, $\text{peak}_\text{min}$, $k_x$, $k_y$, \newline \indent \indent $i_\text{dilate}$,$i_\text{erode}$, $t$}
            \For {$\text{row}$ \textbf{in} $0{:}f_\text{bins}$}
                \State $r \gets \text{gaussian\_smooth}(h[\text{row},:], \sigma)$
                \State $p \gets \text{find\_peaks}(r, \text{peak}_\text{min})$
                \State $\text{peaks}[\text{row}, p] \gets 1$
            \EndFor
            \State $\text{kernel} = \text{cross\_kernel}(k_x, k_y)$
            \State $d \gets \text{dilate}(\text{peaks}, i_{\text{dilate}}, \text{kernel})$
            \State $e \gets \text{erode}(d, i_{\text{erode}}, \text{kernel})$
            \State $\text{contours} \gets \text{find\_contours}(e)$
            \For {$i$ \textbf{in} $0{:}\text{contours}$}
                \If {$\text{size}(\text{contours}[i]) > t$}
                    \State $l \gets \text{draw\_contour}(l, \text{contours}[i], i+1)$
                \EndIf
            \EndFor
            \State \Return $l$
        \EndFunction
    \end{algorithmic}
\end{algorithm}

\begin{figure*}
    % %\includegraphics[width=0.5\linewidth]{curves_edt.png}%
    % %\includegraphics[width=0.5\linewidth]{curves_diffusion.png}
    % \includegraphics[width=\linewidth]{ridges_gauss+edt-bone_region3.pdf}
    % \james{will produce an overall plot with probabilities}
    % \caption{Detected ridges from the combined field histogram.}
    % \label{fig:curves}
  \centering
  \includegraphics[width=.9\textwidth]{compute_probabilities_gauss+edt_bone_region3}\\
  \includegraphics[width=.9\textwidth]{hist_slice_gauss+edt_bone_region3}\\
  \caption{
    (a) Measured 2D histogram for the compound diffusion+distance field from
    Step 2. (b) The detected ridges from Step 3. (c) Sum of computed frequency
    distribution models with optimized smooth piecewise cubic functions for
    parameters $a(\fval), b(\fval), c(\fval), d(\fval)$: this is the part of
    the 2D histogram explained by the smooth model. (d) The individual
    distributions are evaluated on the 2D field-value$\times$ voxel-value grid.
    (e) A single 1D slice of the 2D histogram (black) together with the model
    frequency distributions. Red is soft tissue and yellow is bone mineral.
  }
  \label{fig:curves-and-more}
\end{figure*}

\vspace{\baselineskip}
\noindent\textbf{Step 4: Compute models for material probability distributions} \\
Our goal is to obtain good conditional probability distributions
$\Pof{m}{v,\xx}$ that model the likelihood of a voxel having material type $m$
as a function both on its value $v$ and its position $\xx$ in space. For the
probabilities conditioned on the field values (distance or diffusion field), we
model the probability distributions $\Pof{m}{v,f(\xx)}$ conditioned on voxel-
and field values. We want to make sure that these distribution functions vary
smoothly across space (or as a function of field values), to ensure that we can
identify the materials correctly across the entire image: i.e., even though the
frequency distributions look completely different close to the titanium implant
compared to the middle region or sample surface, we can track the unbroken,
smooth deformation to assign a global material identity.

To this end, we first {\it model the frequency distributions} using the 2D
histograms. Given that we are modeling materials $m=1,\ldots,M$, we write the
full 2D histogram as a sum of distributions $g_m(\fval,v)$ representing the
modeled part, and a residual $r(\xx,v)$.
\begin{equation}
  \label{eq:hist}
  H(\fval,v) = \sum_{m=1}^M g_m(\fval,v) + r(\fval,v)
\end{equation}
The residual is constrained to be non-negative, i.e., we must not explain more
voxels than the image contains. The distribution functions can be chosen in any
way that approximately model the observed frequencies: we first used Gaussians
with passable success, but found that they dropped off too rapidly. We instead
found excellent results with the next-simplest model, leaving the exponential
power $d_m(x)$ as a free parameter:
\begin{equation}
  \label{eq:dist-form}
  g_m(\fval,v) = a_m(\fval) e^{-b_m(\fval) \vert v-c_m(\fval)\vert^{d_m(\fval)}}
\end{equation}
In Eq.~\eqref{eq:dist-form}, each field value $\fval$ $a(\fval)$ is the
distribution height at the center $v=c(\fval)$; $b(\fval)$ is the exponential
falloff rate; and $d(\fval)$ is the exponential power ($d=2$ yields a Gaussian,
$d=1$ a simple exponential). In practice, we found $1.5\le d \le 2$ to best
match the actual frequency distribution decay rates.

Using the ridges found in the previous step, we generate a good starting
guesses and constraints for the distribution parameters $a,b,c,d$: For each
field-value $\fval$ (corresponding to a row in the 2D histogram), we initialize
the starting approximation as:
\begin{equation}
  \label{eq:starting-guesses}
  \begin{array}{lll}
    c_m(\fval) &= \mathop{\mathtt{argmax}}_{v \text{ with }\lab[\fval,v] = m} H(\fval,v)  & \text{Peak position}  \\
    a_m(\fval) &= H(\fval,c_m(\fval)) & \text{Peak value}\\
    b_m(\fval) &= 3/\mathrm{width}_m(\fval)^2  & \text{Decay rate}\\
    d_m(\fval) &= 2 & \text{Exponential power}
  \end{array}
\end{equation}
where we use half the distance to the center of ridge $m+1$ as the width
$\mathrm{width_m}(\fval)$, using the relation that $b = 3/w^d$ yields a $5\%$
cutoff at $w$ for any $1\le d \le 2$. This approximation already yields a good
approximation. Thus, the subsequent optimization using the constrained
quasi-Newton optimization method L-BFGS-B\cite{BFGS} converges rapidly to an
excellent fit. Each 1D histogram row is first optimized independently in
parallel: The resulting numerical functions $a_m(\fval),\ldots,d_m(\fval)$ are
then converted into piecewise cubic functions using a least squares-based
algorithm that ensures continuity and differentiability across the piecewise
segments. This lets us interpolate across outliers due to noise, but equally
important: extrapolate our models smoothly into the regions very close to the
implant, where we don't have enough voxels to produce good statistics.

We finally obtain the {\it conditional probabilities} from the material
frequency distribution models $g_m$ as:
\begin{equation}
  \label{eq:Pm}
  \Pof{m}{v,f(\xx)} = \frac{g_m(v,f(\xx))}{H(v,f(\xx))}
\end{equation}
(well-defined where $H(v,\fval) > 0$, zero outside this region as $0\le g_m \le
H$).


\vspace{\baselineskip}
\noindent\textbf{Step 5: Perform segmentation} \\
The final step of the segmentation process is to apply these probabilities for
segmentation. For each voxel, we obtain its voxel value and field value, which
are then used to perform a lookup in the probability distributions, giving us a
probability that the voxel belongs to a certain class. The voxel is then
assigned the class with the highest probability, meaning we have effectively
classified the voxel based on its voxel intensity and spatial position. In
addition, the segmentation confidence is quantified by the probability.

\begin{algorithm}
    \caption{Final segmentation from the probability distributions.}
    \label{alg:segment}
    \begin{algorithmic}
        \Function {Segment} {$\text{voxels}[n_z,n_y,n_x], p[n_\text{classes},f_\text{bins},v_\text{bins}],$ \newline \indent \indent $v_\text{bins}, v_\text{min}, v_\text{max}, f_\text{bins}, f_\text{min}, f_\text{max}$}
            \For {$z,y,x$ \textbf{in} $0{:}n_z,0{:}n_y,0{:}n_x$}
                \State $v \gets \text{voxels}[z,y,x]$
                \If {$v_\text{min} \leq v \leq v_\text{max}$}
                    \State $f \gets \text{voxels}[z,y,x]$
                    \If {$f_\text{min} \leq f \leq f_\text{max}$}
                        \State $v_i \gets (v_\text{bins} - 1) - \frac{v - v_\text{min}}{v_\text{max} - v_\text{min}}$
                        \State $f_i \gets (f_\text{bins} - 1) - \frac{f - f_\text{min}}{f_\text{max} - f_\text{min}}$
                        \State $\text{probabilities} \gets p[:,f_i,v_i]$
                        \State $\text{result}[z,y,x] \gets \text{argmax}(\text{probabilities}) + 1$
                    \EndIf
                \EndIf
            \EndFor
            \State \Return $\text{result}$
        \EndFunction
    \end{algorithmic}
\end{algorithm}

\subsection{Results of the field-segmentation}

Here we present the final output from the steps described above.

\begin{figure}
    \centering
    \begin{subfigure}[b]{\linewidth}
    \centering
        \includegraphics[width=.7\linewidth]{figures/blood_old_bone_100.png}
        % TODO opdater hvis en anden slice størrelse bliver brugt. voxel size = 3.75
        \caption{A $375\mu m \times 4230\mu m \times 6480\mu m$ slice of the blood network in the old bone region.}
        \label{fig:blood-old-slice}
    \end{subfigure}
    \begin{subfigure}[b]{\linewidth}
    \centering
        \includegraphics[width=.7\linewidth]{figures/blood_new_bone_100.png}
        \caption{A $375\mu m \times 4230\mu m \times 6480\mu m$ slice of the blood network in the new bone region.}
        \label{fig:blood-new-slice}
    \end{subfigure}
    \begin{subfigure}[b]{.48\linewidth}
    \centering
        \includegraphics[width=.9\linewidth,height=\linewidth]{figures/blood_old_cube.png}
        \caption{A $1mm \times 1 mm \times 1 mm$ cube of the blood network in the old bone region.}
        \label{fig:blood-old-cube}
    \end{subfigure}
    \hfill
    \begin{subfigure}[b]{.48\linewidth}
    \centering
        \includegraphics[width=.9\linewidth,height=\linewidth]{figures/blood_new_cube.png}
        \caption{A $1mm \times 1 mm \times 1 mm$ cube of the blood network in the new bone region.}
        \label{fig:blood-new-cube}
    \end{subfigure}
    \caption{
        3D renders of the blood network. Note the difference between the
        capillary network in the old bone region
        (\ref{fig:blood-old-slice},\ref{fig:blood-old-cube}) compared to the
        newly grown bone region
        (\ref{fig:blood-new-slice},\ref{fig:blood-new-cube}).
    }
    \label{fig:blood-network}
\end{figure}

\subsubsection{Sub-classification of soft tissue}

With a good separation of soft tissue and bone, we map out the blood vessel
network using connected components analysis, which is visualized in the 3D
renders in~\Cref{fig:blood-network}. Here we see that we have successfully
segmented the blood vessels out of the bone region. It is especially prominent
when looking at the capillary network, as we can see these in fine detail.
Noteworthy, the newly formed bone region (\Cref{fig:blood-new-slice}) contains
larger concentrations of small blood vessels, compared to the old bone
(\Cref{fig:blood-old-slice}). If we zoom in on a small cube region
(\Cref{fig:blood-old-cube} and \Cref{fig:blood-new-cube}), we see it even more
defined, clearly seeing how the larger vessels connect through the smaller
ones.

The osteocytes are selected by volume and shape: For every connected component
in the feasible volume range, its principal axis and best ellipsoid are computed
and checked against the potential osteocyte shape.

\subsection{Assessing bone-implant contact and blood-implant contact}

\begin{figure}
  \centering
  \begin{tabular}{c}
    \includegraphics[width=\linewidth]{770c_pag-bic-xy-1x} \\
    \includegraphics[width=\linewidth]{770c_pag-bic-P01-xy-1x}
  \end{tabular}
  \caption{
    XY slices of the original tomography (top), and the classified (bottom).
    The voxels are colored according to the modeled probability functions
    $P(m\vert v,\fval)$ between 0 and 1: completely red voxels have $P(m=0\vert
    v,\fval) = 1$, completely yellow voxels have $P(m=1\vert v,\fval)\ = 1$,
    and uncertain voxels become progressively gray.
  }
  \label{fig:histology-comparison1}
\end{figure}

\begin{figure}
  \centering
  \begin{tabular}{c}
    \hspace{-0.5cm}\includegraphics[width=.7\linewidth]{770c_pag-full-P01-yz-1x-gimp} \\
    \includegraphics[width=.7\linewidth]{770c_pag-bic-P01-yz-1x}
  \end{tabular}
  \caption{
    ZY slices of the segmentation as seen far away (top) and zoomed in (bottom).
    Yellow depicts bone and red depicts soft tissue. Note that the segmentation
    correctly classifies the materials close to the implant, even in the
    grooves of the screw threads.
  }
  \label{fig:histology-comparison2}
\end{figure}

Once segmented, tissue in contact with the implant can be studied using the
Euclidean Distance Transform (EDT) from the implant, restricted to the bone
region. We can simply mask the voxels that are within a thin shell of
distances, $d_{min} < d(x,y,z) \le d_{max}$, for example, $d_{min} = 1\mu m$ to
$d_{max} = 5\mu m$. We then sum over the masked voxels of each tissue type to
obtain and divide by the total to obtain the tissue-to-implant contact per area
or study the distribution across the surface area qualitatively.
%TODO: Gør det!

A larger quantitative study is planned that analyses the full data set against
recently conducted histological microscopy taken from the same biopsies. In the
present work, we evaluate qualitatively: Since the SRµCT tomograms are clear
enough that it is possible as humans to distinguish between blood vessels and
bone, as our mammalian visual cortex automatically corrects for the distortion
effects, we can verify the success of the automatic classification.

Figures \ref{fig:histology-comparison1} and \ref{fig:histology-comparison2}
show the same 2D slices as were shown in Figures \ref{fig:3viewsample} and
\ref{fig:slices}, allowing us to visually inspect them side by side. The voxels
are colored according to the segmentation confidence, with the degree of red
being proportional to the modeled probability $\Pof{0}{v,\fval}$ and the degree
of yellow being proportional to $\Pof{1}{v,\fval}$. Grey voxels indicate low
model probabilities of both: either due to the voxel belonging to another
material or simply low computed confidence of the model. By comparing Figures
\ref{fig:histology-comparison1} and \ref{fig:histology-comparison2}, we see
that the computed classification matches the human classification everywhere
where it is possible to visually distinguish the voxels. However, in a thin
1-voxel border to the implant, we cannot verify the segmentation, as the voxel
values are so blended together with the implant voxel values that they become
indistinguishable. A further strengthening of the analysis is needed to reach
this layer. It is possible that the information is irretrievably lost, or
perhaps it can be retrieved through a deconvolution - or simply a more precise
version of the present analysis.

% \subsubsection{Cylindrical projection of implant contact-surface}

% In this subsection, we show how to map the tissue-to-implant contact surface
% for visual inspection. The challenge is that the equidistant region forms a
% curved surface that cannot simply be flattened. Instead, we project the
% voxels onto a cylinder, as shown in Figure \ref{fig:cylinder1}.

% \begin{figure}
%   \centering
%   \begin{tabular}{cc}
%     (a) & \begin{tabular}{c}\includegraphics[width=.8\linewidth]{implant-FoR_prime-circle}\end{tabular} \\
%     (b) & \begin{tabular}{c}\includegraphics[width=0.4\linewidth]{implant-FoR_cylinder-y}%
%   \includegraphics[width=0.5\linewidth]{implant-FoR_cylinder-z2}\end{tabular}
%   \end{tabular}
%   \caption{
%     (a) The implant is divided into 100 segments along its principal axis,
%     and for each segment, the circumscribed circle is computed.
%     (b) The best overall cylindrical fit for the implant is computed using
%     least squares. The implant contact is projected onto this cylinder,
%     and can now be shown in a flat plot.}
%   \label{fig:cylinder1}
% \end{figure}

%%% Local Variables:
%%% mode: latex
%%% TeX-master: "main"
%%% End:
