\section{Method}\label{sec:method}
% Spatial correlation 2d histograms (den nemme, 1 2d hist)
In this section, we will describe the method for exploiting spatial correlation based of the 2D-histograms of the tomographies.

\subsection{1-dimensional histograms}
%diskuter overlappende materialer
If we look at a 1D-histogram of a tomography, as shown in~\Cref{fig:1d-hist}, we see that it is hard to globally threshold which material belongs to which voxel intensity.
This is because the voxel intensity is not a globally defined value, as explained in~\Cref{sec:physics}, which means that the different materials cover ranges of intensities, which blend together in the histogram.
Instead, we should exploit the spatial information in the tomography, as the intensities varies according based of off their positioning.

\begin{figure}
    \centering
    \carl{TODO 1D histogram of a tomography. Skal måske have nogle fordelinger "fittet" ind, så man kan se overlappet?}
    \caption{1-dimensional histogram of a tomography. The x-axis depicts voxel intensity.}
    \label{fig:1d-hist}
\end{figure}

\subsection{Correlation of the different angles}
%Show the correlation of x,y,z,r to histograms 2d
In order to include spatial information, we consider \emph{expanding} a dimension to produce multiple 1-dimensional histograms.
Each histogram is computed by locking the dimension in question and produce
and combine these for each entry in the expanded axis into a new 2-dimensional histogram.
We consider expanding four different axes of the tomography: x, y, z and radius (r).
Each of them can be computed with a single pass over the entire tomography, as the code in~\Cref{lis:2dhists} shows.
%discuss the effects with references to the physics (previous section)

\begin{figure}
    \begin{lstlisting}[language=Python,caption=Python-like pseudo code for computing 2D histograms.,label=lis:2dhists]
for z, y, x in indices:
    v = voxels[z,y,x]
    r = int(sqrt((x-cx)**2 + (y-cy)**2))
    zhist[z,v] += 1
    yhist[y,v] += 1
    xhist[x,v] += 1
    rhist[r,v] += 1
    \end{lstlisting}
\end{figure}

The four expanded histograms can be seen in~\Cref{fig:2dhists}.
We can see that the different expansions divide the materials differently, based on where in the tomography we look.
E.g. the top part of the \texttt{y} histogram only has a single distinquishable material, whereas the bottom part carries multiple clearly distinquished materials.

\begin{figure}
    \centering
    \carl{TODO four plots of the 2d histograms x, y, z, and r.}
    \caption{2D histograms of the different dimension expansions.}
    \label{fig:2dhists}
\end{figure}

\subsection{Removing unnecessary information.}
\carl{Mention how we apply the implant and bone masks? Or should it be before, so that the x,y,z,r histograms are also 2 materials?}

\subsection{EDT, diffusion}
%Show that the fields can perfectly seperate
%discuss all the way to implant contact
While the four histograms further seperate the materials at different angles, they are not good at depicting the materials close to the implant.
\carl{I'm not really sure that's the reason for field histogramming? Isn't it more like for this simple case, it does a well enough job, as the bayesian approach would be the optimal?}

In order to obtain spatial information based on the relation to the implant, we construct two \emph{field} representations: Euclidian Distance Transform (EDT) and Diffusion.

%EDT is good overall, but difficult close to the implant.
EDT computes each voxel as the euclidian distance to the implant. This makes for a generally good representation for all but the voxels that are close to the implant. This is mainly due to the fact that the voxels in the grooves of the implant gets the same value as the voxels next to the threads of the implant, as is shown in~\Cref{fig:edt-vs-diffusion}.

%Use diffusion to "zoom in" on the implant, as it is good close to the implant, but throws everything far away into the same bin
To zoom in on the implant, we utilize diffusion where the implant "glows" further accumulating the value when a voxel is surrounded by implant. This is shown in~\Cref{fig:edt-vs-diffusion}, where the voxels in the grooves are affected by multiple implant voxels.

\begin{figure}
    \centering
    \carl{TODO figure depicting EDT and Diffusion close to the implant, viewed in the xz plane.}
    \caption{Depiction of the different field computations close to the implant in the xz plane. The EDT is depicted in blue, with the diffusion as green.}
    \label{fig:edt-vs-diffusion}
\end{figure}

While Diffusion works well on voxels close to the implant, it assigns the same value to the voxels far from the implant. As such, an optimal solution is to use both the EDT and the diffusion field to better represent both voxels that are far away, as well as the ones that are close to the implant.

%Show that the distance to the implant in edt will be the same in the grooves of the screw compared to the threads of the screw. This is "fixed" with diffusion, as the grooves will be brigher as it is surrounded by more implant.

\subsection{Walkthrough of the method}
This section will describe the steps of the method for going from tomography to material probability distributions.

\subsubsection{Overview}
%Coarse steps, explain the idea
The steps of the segmentation method are:
\begin{enumerate}
    \item Compute the EDT and Diffusion fields to give each voxel spatial information about its relation to the implant.
    \item From the fields, compute 2D histograms to show how the voxel value relates to the field value.
    \item Find the materials within the 2D histograms as curves using image processing techniques.
    \item From the curves, compute probability distributions of each material.
    \item Use these probability distributions to segment each voxel the tomography into the materials.
    \item Project the voxels close to the implant onto an image.
\end{enumerate}
These steps are also summarized in the flow chart in~\Cref{fig:flowchart}.

\begin{figure}
    \centering
    \begin{tikzpicture}
        \node[draw] (field) at (0,0) {Compute fields};
        \node[draw, below = of field] (hists) {Compute 2D histograms};
        \node[draw, below = of hists] (curvs) {Find Curves};
        \node[draw, below = of curvs] (probs) {Compute probabilities};
        \node[draw, below = of probs] (segm) {Segment the voxels};
        \node[draw, below = of segm] (bic) {Project close to implant using BIC};

        \path[draw, ->] (field) -- (hists);
        \path[draw, ->] (hists) -- (curvs);
        \path[draw, ->] (curvs) -- (probs);
        \path[draw, ->] (probs) -- (segm);
        \path[draw, ->] (segm) -- (bic);
    \end{tikzpicture}
    \caption{Flowchart depicting the steps of the method.
    \carl{Skal nok labeles, så man nemt kan se hvilken section der beskriver det. Plus, teksten i kasserne nok også kan være mere flavorful!}
    \carl{Kan måske arrangeres ligesom Davids ting oversigt? Giver måske kun mening hvis der var mere end bare segmentation -> bic}}
    \label{fig:flowchart}
\end{figure}

\subsubsection{Segmentation}
Now we will list each step detailing the actual implementation for segmentation. The steps for segmentation are steps 1-5.
\carl{better flavortext}

\vspace{\baselineskip}
\noindent\textit{\textbf{1. Field computations}}

EDT

For diffusion, rather than solving the diffusion equation, we approximate using multiple convolutions of a 3D-gaussian kernel.
Instead of a single pass with a 3D-kernel, we apply 3 1D convolutions, one in each dimension, to the tomography, obtaining the same result as a regular 3D kernel.
The pseudo code can be seen in~\Cref{lis:diffusion}

\begin{figure}
    \begin{lstlisting}[language=Python,caption=Python-like pseudo code for the diffusion approximation.,label=lis:diffusion]
S = [Ny*Nx, Nx, 1]
N = [Nz, Ny, Nx]
for rep in repitions:
    for dim in dimensions:
        for z, y, x in indices:
            X = [z,y,x]
            is = -min(k, X[dim])
            ie = min(k, N[dim]-X[dim]-1)
            idx = z*S[0] + y*S[1] + x*S[2]
            for i in range(is,ie):
                in_idx = idx + i*S[dim]
                output[idx] +=
                    voxels[in_idx] * kernel[i+k]
    \end{lstlisting}
\end{figure}

%EDT + Diffusion
To obtain good seperation both close to the implant and far away, we combine the two fields into a single one.
\carl{TODO how is this actually done? Or should it just be that both are used during segmentation, each contributing 50 \%? As in, let the probabilities for each of them "take over"}

(remember to show images)

\vspace{\baselineskip}
\noindent\textbf{2D histograms} \\

\vspace{\baselineskip}
\noindent\textbf{Isolate the materials} \\

\vspace{\baselineskip}
\noindent\textbf{Compute the probability distributions} \\

\subsubsection{BIC computations}

\subsection{Results of the field segmentation}

