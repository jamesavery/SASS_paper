\section{Method}\label{sec:method}
% Spatial correlation 2d histograms (den nemme, 1 2d hist)
In this section, we will describe the method for exploiting spatial correlation based of the 2D-histograms of the tomographies. 

\subsection{1-dimensional histograms}
%diskuter overlappende materialer
If we look at a 1D-histogram of a tomography, as shown in~\Cref{fig:1d-hist}, we see that it is hard to globally threshold which material belongs to which voxel intensity. 
This is because the voxel intensity is not a globally defined value, as explained in~\Cref{sec:physics}, which means that the different materials cover ranges of intensities, which blend together in the histogram.
Instead, we should exploit the spatial information in the tomography, as the intensities varies according based of off their positioning.  

\begin{figure}
    \centering
    \carl{TODO 1D histogram of a tomography. Skal måske have nogle fordelinger "fittet" ind, så man kan se overlappet?}
    \caption{1-dimensional histogram of a tomography. The x-axis depicts voxel intensity.}
    \label{fig:1d-hist}
\end{figure}

\subsection{Correlation of the different angles}
%Show the correlation of x,y,z,r to histograms 2d
In order to include spatial information, we consider \emph{expanding} a dimension to produce multiple 1-dimensional histograms.  
Each histogram is computed by locking the dimension in question and produce 
and combine these for each entry in the expanded axis into a new 2-dimensional histogram. 
We consider expanding four different axes of the tomography: x, y, z and radius (r). 
Each of them can be computed with a single pass over the entire tomography, as the code in~\Cref{lis:2dhists} shows. 
%discuss the effects with references to the physics (previous section)

\begin{figure}
    \begin{lstlisting}[language=Python,caption=Python-like pseudo code for computing 2D histograms.,label=lis:2dhists]
for z, y, x in indices:
    v = voxels[z,y,x]
    r = int(sqrt((x-cx)**2 + (y-cy)**2))
    zhist[z,v] += 1
    yhist[y,v] += 1
    xhist[x,v] += 1
    rhist[r,v] += 1
    \end{lstlisting}
\end{figure}

The four expanded histograms can be seen in~\Cref{fig:2dhists}. 
We can see that the different expansions divide the materials differently, based on where in the tomography we look.
E.g. the top part of the \texttt{y} histogram only has a single distinquishable material, whereas the bottom part carries multiple clearly distinquished materials. 

\begin{figure}
    \centering
    \carl{TODO four plots of the 2d histograms x, y, z, and r.}
    \caption{2D histograms of the different dimension expansions.}
    \label{fig:2dhists}
\end{figure}

\subsection{Removing unnecessary information.}
\carl{Mention how we apply the implant and bone masks? Or should it be before, so that the x,y,z,r histograms are also 2 materials?}

\subsection{EDT, diffusion}
%Show that the fields can perfectly seperate
%discuss all the way to implant contact
While the four histograms further seperate the materials at different angles, they are not good at depicting the materials close to the implant.


EDT is good overall, but difficult close to the implant. 

Use diffusion to "zoom in" on the implant, as it is good close to the implant, but throws everything far away into the same bin

Show that the distance to the implant in edt will be the same in the grooves of the screw compared to the threads of the screw. This is "fixed" with diffusion, as the grooves will be brigher as it is surrounded by more implant.

\subsection{Walkthrough of the method}

\subsubsection{Overview}
Coarse steps, explain the idea

steps -> flow chart
\begin{figure}
    \centering
    \begin{tikzpicture}
        \node[draw] (field) at (0,0) {Compute the fields};
        \node[draw, below = of field] (hists) {Compute 2D histograms};
        \node[draw, below = of hists] (curvs) {Find Curves};
        \node[draw, below = of curvs] (probs) {Compute probabilities};
        \node[draw, below = of probs] (segm) {Segment the voxels};
        \node[draw, below = of segm] (bic) {Project close to implant using BIC};

        \path[draw, ->] (field) -- (hists);
        \path[draw, ->] (hists) -- (curvs);
        \path[draw, ->] (curvs) -- (probs);
        \path[draw, ->] (probs) -- (segm);
        \path[draw, ->] (segm) -- (bic);
    \end{tikzpicture}
    \caption{Flowchart depicting the steps of the method. 
    \carl{Skal nok labeles, så man nemt kan se hvilken section der beskriver det. Plus, teksten i kasserne nok også kan være mere flavorful!}
    \carl{Kan måske arrangeres ligesom Davids ting oversigt? Giver måske kun mening hvis der var mere end bare segmentation -> bic}}
    \label{fig:flowchart}
\end{figure}

\subsubsection{Segmentation}

\begin{description}
    \item[Field-computations]

    EDT

    Diffusion

    EDT + Diffusion

    (remember to show images)

    \item[2D histograms]

    \item[Curves]

    \item[Probabilities]
\end{description}

\subsubsection{BIC computations}

\subsection{Results of the field segmentation}

