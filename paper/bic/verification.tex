\section{Verification and analysis}
\label{sec:verification}

In this section, we present the verification of the segmentation method
described in \Cref{sec:method}. First we qualitatively verify that our method
is sound by comparing the segmentation results to the original tomography and
creating an artificial ground truth. Then we quantitatively verify our method
by generating synthetic data, evaluating it on the created ground truth, and
comparing the results to other existing methods. Finally, we analyze the
segmented data to study the blood vessel network in the bone region and produce
statistics on the tissue-to-implant contact.

\subsection{Qualitative verification}

Since the SRµCT tomograms are clear enough that humans can distinguish between
blood vessels and bone, as our mammalian visual cortex automatically corrects
for the distortion effects, we can verify the success of the automatic
classification. We can compare the automatic classification with the manual
classification of the histological microscopy. A larger quantitative study is
planned that analyses the full data set against recently conducted histological
microscopy taken from the same biopsies. As such, we can only present a
qualitative verification of the method in this paper.

Figures \Cref{fig:histology-comparison1} and \Cref{fig:histology-comparison2}
show the same 2D slices as were shown in Figures \Cref{fig:3viewsample} and
\Cref{fig:slices}, allows us to visually inspect them side by side. The voxels
are colored according to the segmentation confidence, with the degree of red
being proportional to the modeled probability $\Pof{0}{v,\fval}$ and the degree
of yellow being proportional to $\Pof{1}{v,\fval}$. Grey voxels indicate low
model probabilities of both: either due to the voxel belonging to another
material or simply low computed confidence of the model. By comparing Figures
\Cref{fig:histology-comparison1} and \Cref{fig:histology-comparison2}, we see
that the computed classification matches the human classification everywhere
where it is possible to visually distinguish the voxels.

\begin{figure*}
    \centering
    \begin{tabular}{cc}
        \includegraphics[width=.45\linewidth]{generated/770c_pag_segmented_yx_raw.pdf} &
        \includegraphics[width=.45\linewidth]{generated/770c_pag_global_yx_otsu.pdf}
        \\
        \includegraphics[width=.45\linewidth]{generated/770c_pag_segmented_yx_colored.pdf} &
        \includegraphics[width=.45\linewidth]{generated/770c_pag_global_yx.pdf} \\
    \end{tabular}
    \caption{
        YX slices of the original tomography (top left), our classification
        (bottom left), and global threshold classification (bottom right). For
        our classification, the voxels are colored according to the modeled
        probability functions $P(m\vert v,\fval)$ between 0 and 1: completely
        red voxels have $P(m=0\vert v,\fval) = 1$, completely yellow voxels
        have $P(m=1\vert v,\fval)\ = 1$, and uncertain voxels become
        progressively gray.
    }
    \label{fig:histology-comparison1}
\end{figure*}

\begin{figure*}
    \centering
    \begin{tabular}{cc}
        \includegraphics[width=.45\linewidth]{generated/770c_pag_segmented_zy_raw.pdf} &
        \includegraphics[width=.45\linewidth]{generated/770c_pag_global_zy_otsu.pdf}
        \\
        \includegraphics[width=.45\linewidth]{generated/770c_pag_segmented_zy_colored.pdf} &
        \includegraphics[width=.45\linewidth]{generated/770c_pag_global_zy.pdf} \\
    \end{tabular}
    \caption{
	    YZ slices of the original tomography (top left), Otsu segmentation
	    (top right), our classification (bottom left), and global threshold
	    classification (bottom right).  Yellow depicts bone and red depicts
	    soft tissue. Note that our segmentation correctly classifies the
	    materials close to the implant, even in the grooves of the screw
	    threads, whereas the global threshold classification fails.
    }
    \label{fig:histology-comparison2}
\end{figure*}

We see that the segmentation is overall successful, classifying the internal
darker regions as soft tissue and the lighter regions as bone. This is further
confirmed in \Cref{sec:blood-network}. However, extremely close to the implant
(a few voxels), we cannot verify the segmentation, as the voxel values are so
blended together with the implant voxel values that they become
indistinguishable. A further strengthening of the analysis is needed to reach
this layer. It is possible that the information is irretrievably lost, or
perhaps it can be retrieved through a deconvolution - or simply a more precise
version of the present analysis. It also fails within some of the soft tissue
blobs, which is due to air bubbles in the resin, which have a very low density
and thus a very low absorption.

\subsection{Quantitative evaluation}

% Why it is a problem
Reliably evaluating segmentation methods on medical data such as that
presented in this paper is a challenging task. A satisfactory annotated ground
truth does not exist, which makes quantitatively evaluating one method against
another very hard. Furthermore, due to the very large number of voxels, using
manual or semi-automatic methods quickly becomes impractical. One such example
are neural networks, which need to be trained on multiple instances of similar
data accompanied by an annotated ground truth. This is a very time consuming
and expensive procedure, and does not easily translate to other data, which are
only slightly dissimilar.

In order to ensure a reliable and consistent verification of our method, we
must tackle a common problem in medical physics. Namely that problems are often
solved using approaches which can be hard to reproduce and therefore
verify~\cite{replication-crisis}. We try to overcome this difficulty by
creating synthetic reproducible data, accompanied by a consistent reference
level ground truth. The generated data resembles our original data and shares
many physical qualities, but instead of being acquired at ESRF, it is simulated
using the software Novi-sim~\cite{novisim} on an affordable workstation using a
consumer-grade GPU (Nvidia RTX 3080). This allows other researchers to not only
reproduce but also modify our tomographic models and perform new experiments.
This attempts to both extend the robustness of our proposed method, and make it
less specific and dependent on our original data. In future experiments this
also allows to extract 2D-slices of synthetic data and compare with manually
inspected and annotated histologies.

Original data and synthetic data, with STL files and ground truth are available
at github link.
% TODO: add link

\subsubsection{Ground truth and synthetic data}

% How we use Novi-sim
The presented segmentation pipeline results in a set of masks, one for each
material.  Each mask is overlayed unto a single volume representing our
artificial ground truth. By construction it will have a perfect overlap,
correctly identifying each material using our method.  Furthermore the masks
for each material are converted into meshes and used to generate STL files.
The generated STL files are transformed to synthetic tomograms using Novisim.
While the software is set up to simulate the original experiment at ESRF, it
will naturally not be a perfect match. The generated ground truth is now
perfectly accurate with the input, but deviates slightly but predictably from
the generated tomogram due to experimental physical effects and artifacts. This
allows us to use reproducible synthetic data to consistently compare our
segmentation method with other, and also detect in which regions they differ
the most. Since we expect regions very close and very far away from the implant
to be difficult, any intermediate regions should be fairly consistent across
methods. Finally we can inspect the differing regions visually to evaluate on
the overall performance.

\subsubsection{Comparison to other methods}

Having a reproducible tomogram with corresponding ground truth, allows us to
make a comparison to other methods. The simplest approach is having a global
threshold. This is used to illustrate why we cannot afford to rely on too
simple methods.  Next we show how a more flexible multi-level automatic Otsu
segmentation performs and its drawbacks. Finally we look at fully automatic
ML-based methods and illustrate the assumptions that underlie their peformance
and how they are not always sufficient or practical for data similar to ours.

\subsection{Dataset analysis}

Having verified the method qualitatively and quantitatively, we can have a more
detailed look into the segmented data. We are interested in the blood vessel
network in the bone region and the tissue-to-implant contact. We will start by
analyzing the blood vessel network and then the tissue-to-implant contact.

\subsubsection{Blood vessel network}
\label{sec:blood-network}

With a good separation of soft tissue and bone, we map out the blood vessel
network using connected components analysis, which is visualized in the 3D
renders in~\Cref{fig:blood-network}. Here we see that we have successfully
segmented the blood vessels out of the bone region. It is especially prominent
when looking at the capillary network, as we can see these in fine detail.
Noteworthy, the newly formed bone region (\Cref{fig:blood-new-slice}) contains
larger concentrations of small blood vessels, compared to the old bone
(\Cref{fig:blood-old-slice}). If we zoom in on a small cube region
(\Cref{fig:blood-old-cube} and \Cref{fig:blood-new-cube}), we see it even more
defined, clearly seeing how the larger vessels connect through the smaller
ones.

\begin{figure}
    \centering
    \begin{subfigure}[b]{\linewidth}
    \centering
        \includegraphics[width=.7\linewidth]{generated/figure10_old_bone.png}
        % TODO opdater hvis en anden slice størrelse bliver brugt. voxel size = 3.75
        \caption{A 375 µm $\times$ 4230 µm $\times$ 6480 µm slice of the blood network in the old bone region.}
        \label{fig:blood-old-slice}
    \end{subfigure}
    \begin{subfigure}[b]{\linewidth}
    \centering
        \includegraphics[width=.7\linewidth]{generated/figure10_new_bone.png}
        \caption{A 375 µm $\times$ 4230 µm $\times$ 6480 µm slice of the blood network in the new bone region.}
        \label{fig:blood-new-slice}
    \end{subfigure}
    \begin{subfigure}[b]{.48\linewidth}
    \centering
        \includegraphics[width=.9\linewidth,height=\linewidth]{generated/figure10_old_cube.png}
        \caption{A $1mm \times 1 mm \times 1 mm$ cube of the blood network in the old bone region.}
        \label{fig:blood-old-cube}
    \end{subfigure}
    \hfill
    \begin{subfigure}[b]{.48\linewidth}
    \centering
        \includegraphics[width=.9\linewidth,height=\linewidth]{generated/figure10_new_cube.png}
        \caption{A $1mm \times 1 mm \times 1 mm$ cube of the blood network in the new bone region.}
        \label{fig:blood-new-cube}
    \end{subfigure}
    \caption{
        3D renders of the blood network. We see a difference between the
        capillary network in the old bone region
        (\ref{fig:blood-old-slice},\ref{fig:blood-old-cube}) compared to the
        newly grown bone region
        (\ref{fig:blood-new-slice},\ref{fig:blood-new-cube}), where the new
        bone region contains a higher concentration of small blood vessels,
        compared to the old bone region containing fewer larger blood vessels.
    }
    \label{fig:blood-network}
\end{figure}

\subsubsection{Implant contact}
\label{sec:contact}

Tissue in contact with the implant can be studied using the EDT from the
implant, restricted to the bone region. We can simply mask the voxels that are
within a thin shell of distances, $d_{min} < d(x,y,z) \le d_{max}$, for
example, $d_{min} = 1 \text{µm}$ to $d_{max} = 5 \text{µm}$. We then sum over
the masked voxels of each tissue type to obtain and divide by the total to
obtain the tissue-to-implant contact per area or study the distribution across
the surface area qualitatively. In particular, we are interested in the
difference between the old and new bone regions. We define the metric as
follows:

\begin{equation}
    BIC = \frac{\sum_{i : \text{field}(i) < 5} \text{bone}(i)}{\sum_{i : \text{field}(i) < 5} \text{voxels}(i)}
\end{equation}

We find the old and new bone regions at different $z$ levels, based on the
threading of the implant, with new bone being near the small threads and old
bone being near the large threads. The results can be seen in \Cref{tab:bic}.
We see that the BIC is higher in the old bone region than in the new bone
region, which is expected as the bone grows around the implant over time. This
is further confirmed by the blood vessel network analysis, as the new bone
region contains a higher concentration of small blood vessels, indicating that
the bone is still growing. This shows that the method is successful in
segmenting the bone and soft tissue regions and that the segmentation can be
used to study BIC.

\begin{table}
    \caption{The mean and standard deviation of bone-to-implant contact in the old and new bone regions for each of our samples.}
    \label{tab:bic}
    \centering
    \begin{tabular}{lcc}
        \toprule
        Sample & Old bone & New bone \\
        \midrule
        770\_pag & $0.5020 \pm 0.1222$ & - \\
        770c\_pag & $0.6519 \pm 0.1055$ & $0.5638 \pm 0.0907$ \\
        771c\_pag & $0.4517 \pm 0.1708$ & $0.5016 \pm 0.1078$ \\
        772\_pag & $0.7012 \pm 0.1065$ & $0.5222 \pm 0.0869$ \\
        775c\_pag & $0.2802 \pm 0.0928$ & $0.2652 \pm 0.1054$ \\
        810c\_pag & $0.6819 \pm 0.0866$ & $0.5155 \pm 0.0897$ \\
        811\_pag & $0.5832 \pm 0.1015$ & $0.3715 \pm 0.0935$ \\
        \bottomrule
    \end{tabular}
\end{table}

