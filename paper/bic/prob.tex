\newcommand{\xx}{\mathbf{x}}
% \newcommand{\fval}{f(\xx)}
% \newcommand{\fval}{f}
\newcommand{\fval}{x}
\newcommand{\lab}{\mathrm{L}}
Our goal is to obtain good conditional probability distributions $P(m|v,\xx)$
that model the likelihood of a voxel having material type $m$ as a function
both on its value $v$ and its position $\xx$ in space. For the probabilities
conditioned on the field values (distance or diffusion field), we model
the probability distributions $P(m|v,f(\xx))$ conditioned on
voxel- and field values. We want to make sure that these distribution
functions vary smoothly across space (or as a function of field values),
to ensure that we can identify the materials correctly across the entire
image: i.e., even though the frequency distributions look completely different
close to the titanium implant compared to the middle region or sample surface,
we can track the unbroken, smooth deformation to assign a global material
identity.

To this end, we first {\em model the frequency distributions} using the
2D histograms. Given that we are modeling materials $m=1,\ldots,M$,
we write the full 2D histogram as a sum of distributions
$g_m(\fval,v)$ modeling the described part, and an undescribed
residual $r(\xx,v)$.
\begin{equation}
  \label{eq:hist}
  H(\fval,v) = \sum_{m=1}^M g_m(\fval,v) + r(\fval,v)
\end{equation}
The residual is constrained to be non-negative, i.e., we must not explain
more voxels than the image contains.
The distribution functions can be chosen in any way that approximately
model the observed frequencies: we first used Gaussians with passable success,
but found that they dropped off too rapidly. We instead found excellent results
with the next-simplest model, leaving the exponential power free:
\begin{equation}
  \label{eq:dist-form}
  g_m(\fval,v) = a_m(\fval) e^{-b_m(\fval) |v-c_m(\fval)|^d_m(\fval)
\end{equation}
where for each field value $\fval$
$a(\fval)$ is the distribution height at the center $v=c(\fval)$,
$b(\fval)$ is the exponential falloff rate, and $d(\fval)$ is
the exponential power ($d=2$ yields a Gaussian, $d=1$ a simple exponential).
In practice we found $1.5\le d \le 2$ to best match the actual frequency
distribution decay rates.

Using the ridges found in Section \ref{sec:ridges}, we generate good starting
guesses and constraints for the distribution parameters
$a,b,c,d$:
For each field-value $\fval$ (corresponding to a row in the 2D histogram),
we initialize the starting approximation as:
\begin{equation}
  \label{eq:starting-guesses}
  \begin{split}
    c_m(\fval) &= \mathop{\mathtt{argmax}}_{v \text{ with }\lab[\fval,v] = m} H(\fval,v)    \\    
    a_m(\fval) &= H(\fval,c_m(\fval))\\
    b_m(\fval) &= 3/\mathrm{width}_m(\fval)^2 \\
    d_m(\fval) &= 2
  \end{split}
\end{equation}
where we use half the distance to the center of ridge $m+1$ as the width
$\mathrm{width_m}(\fval)$, using the relation that $b = 3/w^d$ yields
a $5\percent$ cutoff at $w$ when $1\le d \le 2$. This approximation
already yields a good approximation. Thus, the subsequent
optimization using the constrained quasi-Newton optimization method L-BFGS-B\cite{BFGS}
converges rapidly to an excellent fit. Each 1D histogram row is first optimized
independently in parallel: The resulting numerical functions
$a_m(\fval),\ldots,d_m(\fval)$ are then converted into piecewise cubic
functions using a least squares-based algorithm that ensures continuity
and differentiability across the piecewise segments. This lets us interpolate across
outliers due to noise, but equally important:
extrapolate our models smoothly into the regions very close to the implant, where
we don't have enough voxels to produce good statistics.

We finally obtain the {\em conditional probabilities} from the material frequency distribution
models $g_m$ as:
\begin{equation}
  \label{eq:Pm}
  P(m|v,f(\xx)) = \frac{g_m(v,f(\xx))}{H(v,f(\xx))} 
\end{equation}
(well-defined where $H(v,\fval) > 0$, zero outside this region as $0\le g_m \le H$).



%%% Local Variables:
%%% mode: latex
%%% TeX-master: "main"
%%% End:
