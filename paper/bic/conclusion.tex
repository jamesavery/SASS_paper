\section{Conclusion and future work}
\label{sec:conclusion}

While SRµCT yields 3D reconstructions of extremely high quality compared to
lab-scale X-ray setups, several distortive effects remain that obstruct
accurate tissue classification in important regions, in particular near and at
interfaces of high-contrast transitions such as where biological tissue meets
metallic implants. However, these effects are well-behaved, in the sense that
they vary smoothly over space, making it possible to discover approximate
mathematical models of the effects and counter their resulting distortion.

We were able to build probabilistic models for the distortive effects of soft
tissue and bone voxel values as functions of distance to the implant and as
functions of an approximate diffusion field. This made it possible to see close
to the implant surface, and automatically segment into tissue types throughout
the sample and all the way up to the implant surface, with high accuracy both
at long and short distances.

In this pilot work, we have only looked at utilizing the combined field and
distance transform to the implant surface. We have not yet compared the results
to histology microscopy, which is the gold standard for tissue classification.
However, the results are promising: the segmentation is successful, and the
tissue-to-implant contact can be studied quantitatively and qualitatively. The
method is also general and can be applied to other types of samples with
similar distortion effects.

In upcoming work, we plan a larger quantitative study that compares with
histology microscopy results, obtained from the same biopsies. In addition, we
are extending the method with a Bayesian combination of multiple ``angles'':
different sources of distortion effects (e.g. multiple physical effects) can be
better captured by different fields, e.g. implant-glow may be best captured
by the distance transform to the sample surface, while the diffusion field
captures distortion near metallic implants. Thus, different fields separate
tissue material distributions in different regions and are expected to yield a
stronger analysis when combined. It is hoped that this will make it possible to
separate multiple highly overlapping frequency distributions.

\section{Data availability}

Original data and synthetic data, with STL files and ground truth have been
made freely available. % TODO (James): ERDA eller?

\section{Acknowledgements}

This project has received funding from the European Union’s Horizon 2020
research and innovation programme under grant agreement No.~779322
(``MAXIBONE'').  JA was partially funded by VILLUM FONDEN (VILLUM Experiment
no.~23321, “Folding Carbon: A Calculus of Molecular Origami”).  CJ was partially
funded by the Innovation Fund Denmark (IFD) under File No.~8057-00012B, the IFD
Grand Solutions project ``Adaptive X-ray InSpection''.  JA, AT and CJ was funded
by the VILLUM Foundation through (VILLUM Experiment no.~41017, “OsteoMorph:
Understanding Bone Microphysics through Computational Morphology of SRuCT”).

\section{Conflict of Interest}
On behalf of all authors, the corresponding author states that there is no
conflict of interest.

%%% Local Variables:
%%% mode: latex
%%% TeX-master: "main"
%%% End:
