\section{Conclusion and future work}
\label{sec:conclusion}

While SR$\mu$CT yields 3D reconstructions of extremely high fidelity compared to lab-grade X-ray setups,
several distortion effects remain that obstruct accurate tissue classification in important regions, in particular
near and at interfaces of high-contrast transitions such as where biological tissue meets metallic implants.
However, these effects are well behaved, in the sense that they vary smoothly over space, making it possible
to discover approximate mathematical models of the effects, and counter their resulting distortion.

We were able to build probabilistic models for the distortive effects
of soft tissue and bone voxel values as functions of distance to the
implant, and as functions of an approximate diffusion field. This made
it possible to see all the way up to the implant surface, and
automatically segment into tissue types throughout the sample and all
the way up to the implant surface, with high accuracy both at long and
short distances.

In this pilot work, we have only made a qualitative study. In upcoming
work, we plan a larger quantitative study that compares with histology
microscopy results, obtained from the same biopsies.  In addition, we
are extending the method with Bayesian combination of multiple
``angles'': different sources of distortion effects (e.g.~multiple
physical effects) can be better captured by different fields,
e.g.~beam hardening may be best captured by the distance transform to
the sample surface, while the diffusion field captures distortion near
metallic implants. Thus different fields separate tissue material
distributions in different regions, and in combination are expected to
yield a stronger analysis. It is hoped that this will make it possible
to separate multiple highly overlapping frequency distributions.




\section{Acknowledgements}

This project has received funding from the European Union’s Horizon 2020 research and innovation programme under grant agreement No. 779322 (``MAXIBONE'').
JA was partially supported by the VILLUM Foundation through VILLUM Experiment Project: 41017.
CJ was funded by the Innovation Fund Denmark (IFD) under File No. 8057-00012B, the IFD Grand Solutions project ``Adaptive X-ray InSpection''.


%%% Local Variables:
%%% mode: latex
%%% TeX-master: "main"
%%% End:
